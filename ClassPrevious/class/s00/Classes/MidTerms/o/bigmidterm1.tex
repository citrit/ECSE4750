\documentclass[twoside]{article}
% Time-stamp: </dept/ecse/graphics-f98/bigmidterm.tex, Tue, 16 Feb 1999, 18:38:23 EST, wrf@mab.ecse.rpi.edu>

% Time-stamp: </dept/ecse/graphics/header.tex, Tue, 21 Apr 1998, 19:39:30 PDT, wrf@speed.ecse.rpi.edu>

\listfiles
%%%%%%%% PACKAGES

%\usepackage{alltt}
\usepackage{calc}
%\usepackage{citesort}
%\usepackage[draft]{epsfig}
\usepackage{fancybox}
\usepackage{fancyheadings}
\usepackage[dvips]{graphicx}
\usepackage{epsfig}   % after graphics/graphicx
%\usepackage{floatfig}
% \usepackage{html}
%\usepackage{latexsym}
\usepackage{lgrind}
\usepackage{moreverb}
\usepackage{multicol}
%\usepackage{newcent}
%\usepackage{pifont}
\usepackage{psboxit}
\usepackage{times}
\usepackage{url}
\usepackage{varioref}
% amstex package not on RCS.  Is in CS at /usr/local/packages/tex/doc/amslatex.

%%%%%%% LENGTHS

\setlength{\textwidth}{6.2in}
\setlength{\textheight}{8.8in}
\setlength{\oddsidemargin}{0.25in}
\setlength{\evensidemargin}{0.25in}
\setlength{\topmargin}{-.5in}

\newcommand{\thissemester}{Spring 1998}


\setlength{\textwidth}{6.2in}
\setlength{\textheight}{8.3in}
\setlength{\oddsidemargin}{0.2in}
\setlength{\evensidemargin}{0.2in}
\setlength{\topmargin}{-.3in}

% ********* UPDATE EACH TIME:
\newcommand{\handout}{bigmidterm}
\renewcommand{\thepage}{\handout--\arabic{page}}  % Mitt p 446b

\pagestyle{fancy}
\rhead[ECSE-4750 Computer Graphics]{\thepage}
\chead{Handout \handout}
\lhead[\thepage]{ECSE-4750 Computer Graphics}
\rfoot[Rensselaer Polytechnic Institute]{\thissemester}
\lfoot[\thissemester]{Rensselaer Polytechnic Institute}
\cfoot{}
\setlength{\headrulewidth}{.1mm}
\setlength{\footrulewidth}{.1mm}

%\psdraft


\input{wrf}   % After the other files since it may redefine things.


\begin{document}
%\initfloatingfigs
\PScommands

% \layout

\title{
\emph{This is not one exam!  It is a collection of all the
questions that I've asked on midterm exams over the last several
years.  Some questions refer to material that is not in this
year's course.}\\
% \ovalbox{\hspace*{.8in}\rule{0in}{.4in}} \hspace{1in} Name:
\_\_\_\_\_\_\_\_\_\_\_\_\_\_\_\_\_\_\_\_\_\_\_\_\_\_\_\_\_\_\_\_\_\_\_\_\_\_\_\_ \\[.2in]
ECSE-4750 Computer Graphics,  \thissemester \\
Handout \handout\footnote{Copyright 1996-8, Wm. Randolph Franklin.
You may copy this for nonprofit research and teaching purposes
under these conditions:  (1) You do not change the document.  (2) You do not
charge people to look at or to copy your copy.}
\\
Midterm Exam}
\author{}
% \date{Feb 26, 1997}



\maketitle

%\tableofcontents
%\listoffigures


% \newcounter{points}
% \setcounter{points}{0}

% \newcommand{\ite}[1]{\addtocounter{points}{#1} %
% \item \ovalbox{\hspace*{2em}\rule{0cm}{2em}}/{\tt [#1]\ }}

\newcounter{questionnumber}   %  Remembers the question number across exam sections.

% asp{4}:  Leave 4 inches for the answer.
% \newcommand{\asp}[1]{\vspace*{#1in}}
% But not on the sample version.
\newcommand{\asp}[1]{}

% \enlargethispage*{1in}

\section*{Exam Rules}
\begin{enumerate}
\setlength{\itemsep}{0pt}
\item You may have one 2-sided 8.5 inch x11 inch note sheet, which may be
mechanically printed.  Keep your cribsheet since you can use it again on
the final.
\item You may have your blank paper, calculator, pens, etc.
\item You may not communicate with anyone, except Sutha or me.
\item Answer all questions.  Brief, concise, answers are preferred.
\item Spend time on a question proportional to its number of points.
\item Note that the last page is number  \pageref{end}.
\item Start immediately.  You have until 1:50.
\item Try to write legibly.
\item Write your NAME on top of this page.
\item Try to write your answers on these question sheets, tho extra
paper is allowed.  If an answer is on an extra sheet, say so in
the normal space on this sheet.
\item Leave the small oval boxes blank; they're for our grade.
\item \emph{Warning:}  Be careful of questions that appear to be identical to ones that you've seen before.  Something might have changed.
\end{enumerate}

\section*{Exam}

\begin{enumerate}


\ite{4}  {\tt (points)} Name four things that \verb+XtVaAppInitialize+ does.
\asp{2}

\item Tell me about classes of X programs.
\begin{enumerate}
\ite{1} \texttt{(point)} Where is a program's class set?
\asp{.5}
\ite{1} What is the biggest use of the class?
\asp{.5}
\ite{1} What's the point of classes in general; i.e., why not just
use program names?
\asp{.5}
\end{enumerate}


\ite{3} Name X 3 resources that may be specified on the command line
when running a program.
\asp{1}

\ite{4} In the following diagram, fill in the blanks with the
following labels:
% \enlargethispage*{.5in}
\emph{motif,
xlib,
xt,
user program}.

\begin{minipage}{6in}
\begin{verbatim}
|----------------------------------------------------------|
|                                                          |
|                                                          |
|                                                          |
|-------------------------|                                |
|                         |                                |
|                         |                                |
|                         |                                |
|---------------------------------------|                  |
|                                       |                  |
|                                       |                  |
|                                       |                  |
|----------------------------------------------------------|
|                                                          |
|                                                          |
|                                                          |
|-----------------------------------------------------------
\end{verbatim}
\end{minipage}


\ite{1}
A widget variable in a program is a pointer to a structure, whose
components are documented.  Why is it bad practice to access these
components directly, instead of via things like \texttt{XtName}?
\asp{1}

\ite{40} Here is an alphabetical list of names that occur in X,
including routines, macros, and so on.   Use them to answer the
question starting on \pageref{p:q}.

\begin{multicols}{2}

\begin{enumerate}
\setlength{\itemsep}{0in}
\renewcommand{\theenumii}{\arabic{enumii}}
\item argc
\item argv
\item Buttonpress
\item ButtonPressMask
\item callback
\item CancelButtonCallback
\item CapRound
\item cascade button
\item Class
\item ClearButtonCallBack
\item Coloredit
\item colormap
\item colorpanel
\item Colors
\item DefaultRootWindow
\item Draw
\item editres
\item Event
\item EventMask
\item Expose
\item ExposureMask
\item GC
\item JoinBevel
\item Keypress
\item I18N
\item LineDoubleDash
\item Manage
\item Map
\item OkButtonCallback
\item OSF
\item PrintButtonCallBack
\item PushButton
\item Redraw
\item RowColumn
\item Widget
\item XAllocColor
\item XBlackPixel
\item XButtonEvent
\item XColor
\item XCreateColormap
\item XCreateGC
\item XCreateSimpleWindow
\item XDefaultColormap
\item XDefaultScreen
\item XDisplayHeight
\item XDisplayHeightMM
\item XDrawLine
\item XDrawString
\item XEvent
\item XExam
\item XExposeEvent
\item XFarewell
\item XFillArc
\item XFillPolygon
\item XKeyEvent
\item Xlib
\item XLookupString
\item Xm
\item XMapRaised
\item XmATTACH
\item XmBulletinBoard
\item xmCascadeButtonWidgetClass
\item XmCreateInformationDialog
\item XmCreateMenuBar
\item XmCreatePromptDialog
\item XmCreatePulldownMenu
\item XmCreatePushButton
\item XmDIALOG
\item XmFONTLIST
\item xmFormWidgetClass
\item XmLabel
\item xmLabelWidgetClass
\item XmMainWindowSetAreas
\item xmMainWindowWidgetClass
\item XmMessageBoxGetChild
\item XmNactivateCallback
\item XmNcancelCallback
\item XmNdragCallback
\item XmNeditMode
\item XmNheight
\item XmNlabelString
\item XmNleftAttachment
\item XmNmaximum
\item XmNmenuHelpWidget
\item XmNminimum
\item XmNokCallback
\item XmNsubMenuId
\item XmNtextString
\item XmNvalue
\item XmNvalueChangedCallback
\item XmNwidth
\item XmNwordWrap
\item XmNx
\item XmNy
\item XMove
\item xmRowColumnWidgetClass
\item XmScale
\item XmScaleCallbackStruct
\item xmScaleWidgetClass
\item XmSelectionBoxCallbackStruct
\item xmSeparatorWidgetClass
\item XmString
\item XmStringConcat
\item XmStringCreate
\item XmStringCreateSimple
\item XmStringFree
\item XmTextGetString
\item XmTextSetString
\item xmTextWidgetClass
\item XNextEvent
\item XOpenDisplay
\item XPoint
\item XQueryColor
\item XRootWindow
\item XSelectInput
\item XServerVendor
\item XSetBackground
\item XSetForeground
\item XSetLineAttributes
\item XSetWindowColormap
\item XStoreColor
\item Xt
\item XtActionProc
\item XtActionsRec
\item XtAddCallback
\item XtAddEventHandler
\item XtAppAddActions
\item XtAppContext
\item XtAppMainLoop
\item XtCreateManagedWidget
\item XtCreateWidget
\item XtDestroyWidget
\item XtDisplay
\item XText
\item XtFree
\item XtGrabNone
\item XtMalloc
\item XtManageChild
\item XtName
\item XtNbackground
\item XtNumber
\item XtOverrideTranslations
\item XtParent
\item XtParseTranslationTable
\item XtPointer
\item XtPopup
\item XtRealizeWidget
\item XtSetArg
\item XtSetValues
\item XtUnmanageChild
\item XtVaAppInitialize
\item XtVaCreateManagedWidget
\item XtVaGetValues
\item XtVaSetValues
\item XtWindow
\item XWhitePixel
\end{enumerate}
\end{multicols}

\noindent\label{p:q} Here is a list of X-related operations to perform,
resources, data types, concepts, etc.
Before each item, write down the item NUMBER in the above
list that is most applicable.
\newcommand{\itemsp}{\item \rule{1in}{.01in}}
\begin{enumerate}
\renewcommand{\theenumii}{\arabic{enumii}}
\itemsp A button designed for calling up a pulldown menu.
\itemsp A data type containing graphics properties of a line, such as color, width, and style.
\itemsp A data type of a string of text including font info.
\itemsp A data type whose components are \emph{x} and \emph{y}.
\itemsp A particular program that queries and changes another program's resources.
\itemsp A resource name to cause two widgets to be positioned side-by-side.
\itemsp A resource to tie a pulldown menu to the button that will make it appear.
\itemsp A struct containing components like red, green, blue, and pixel.
\itemsp A widget designed to take several children, which it will lay out itself without any more user control.
\itemsp Converts a Motif string into a C string.
\itemsp Copies a C variable into a widget's resource value.
\itemsp Copies a widget's resource value into a C variable.
\itemsp Creates a popup convenience widget to ask the user for a small amount of input.
\itemsp Creates a widget and tells its parent to determine its size and position.
\itemsp Creates a window for a widget that doesn't yet have one.
\itemsp Creates the top level widget.
\itemsp Deletes a widget.
\itemsp Determines a widget's size and location, and does the same for other widgets that have this widget as a parent.
\itemsp Draws a line.
\itemsp Draws text at the Xlib level.
\itemsp Find the row of the color map containing the color white.
\itemsp For a graphics context, sets the color to draw a solid line in.
\itemsp Find the address of the ok button in a dialog widget.
\itemsp Frees a Motif string that is no longer needed.
\itemsp Gets a widget's name.
\itemsp Gives a user routine to call when a button is activated.
\itemsp Installs a desired color into the color map.
\itemsp Internationalization.
\itemsp Loops while waiting for X events.
\itemsp Processes X arguments on the command line.
\itemsp Reads the resource file.
\itemsp Returns the display associated with a widget.
\itemsp Returns the number of pixels from top to bottom on your display.
\itemsp Returns the widget whose child this widget is.
\itemsp Returns the window associated with a widget.
\itemsp Specifies what classes of events to accept.
\itemsp The industrial consortium behind X.
\itemsp The number of arguments that an X program was run with.
\itemsp Tries to connect to the X server.
\itemsp Waits for the next user input or system status change, like a keypress or window expose.
\end{enumerate}

\ite{8} Here are some lines from \texttt{xfarewell.c}.
Unfortunately, I accidently ran them thru the \texttt{sort}
program, so that they are now in alphabetical order.  Please
place them in the proper order, by writing the proper number of
each line, 1,2,3, or whatever, before that line.   I've numbered
the first three lines.
\begin{enumerate}

\item 1
\begin{verbatim}
#include <Xm/PushB.h>		
\end{verbatim}
\item 2
\begin{verbatim}
#include <Xm/Xm.h>		
\end{verbatim}
\item 3
\begin{verbatim}
#include <stdio.h>		
\end{verbatim}
\itemsp
\begin{verbatim}
XtAppAddActions (app_context, my_actions, XtNumber (my_actions));
\end{verbatim}
\itemsp
\begin{verbatim}
XtAppContext app_context;
\end{verbatim}
\itemsp
\begin{verbatim}
XtAppMainLoop (app_context);}
\end{verbatim}
\itemsp
\begin{verbatim}
XtRealizeWidget (top_level);	
\end{verbatim}
\itemsp
\begin{verbatim}
exit (0);}
\end{verbatim}
\itemsp
\begin{verbatim}
farewell_widget = 
	XtVaCreateManagedWidget ("farewell", xmPushButtonWidgetClass,
	   		         top_level, NULL);
\end{verbatim}
\itemsp
\begin{verbatim}
main (int argc, char **argv)
\end{verbatim}
\itemsp
\begin{verbatim}
static XtActionsRec my_actions[] = {{"confirm", (XtActionProc) Confirm}, 
				    {"quit", (XtActionProc) Quit}};
\end{verbatim}
\itemsp
\begin{verbatim}
top_level = XtVaAppInitialize (&app_context, "XFarewell", NULL, 0,
			       &argc, argv, NULL, NULL);
\end{verbatim}
\itemsp
\begin{verbatim}
void Confirm (Widget w, XButtonEvent * event, String * params, 
	      Cardinal * num_params)
\end{verbatim}
\itemsp
\begin{verbatim}
void Quit (Widget w, XButtonEvent * event, String * params, 
           Cardinal * num_params)
\end{verbatim}
\itemsp
\begin{verbatim}
{Widget top_level, farewell_widget;
\end{verbatim}
\itemsp
\begin{verbatim}
{fprintf (stderr, "Are you sure you want to exit?\n\
Click with the middle pointer button if you're sure.\n");}
\end{verbatim}
\itemsp
\begin{verbatim}
{fprintf (stderr, "It was nice knowing you.\n");
\end{verbatim}
\end{enumerate}

\ite{8} Here are some lines from \texttt{xgoodbye.c}.
Unfortunately, I accidently ran them thru the \texttt{sort}
program, so that they are now in alphabetical order.  Please
place them in the proper order, by writing the proper number of
each line, 1,2,3, or whatever, before that line.   Make variables
local, not global, if possible.
\begin{enumerate}

\itemsp
\begin{verbatim}
#include <Xm/Xm.h>
\end{verbatim}
\itemsp
\begin{verbatim}
goodbye_widget = XtVaCreateManagedWidget("goodbye",
xmPushButtonWidgetClass, top_level_widget, NULL); 
\end{verbatim}
\itemsp
\begin{verbatim}
top_level_widget = XtVaAppInitialize(&app, "XGoodbye", NULL, 0,
&argc, argv, NULL, NULL); 
\end{verbatim}
\itemsp
\begin{verbatim}
void main(int argc, char **argv)
\end{verbatim}
\itemsp
\begin{verbatim}
Widget  top_level_widget, goodbye_widget;
\end{verbatim}
\itemsp
\begin{verbatim}
XtAddCallback(goodbye_widget, XmNactivateCallback, Quit, 0);
\end{verbatim}
\itemsp
\begin{verbatim}
XtAppMainLoop(app);     
\end{verbatim}
\itemsp
\begin{verbatim}
XtRealizeWidget(top_level_widget);
\end{verbatim}
\end{enumerate}

\ite{2} Altho SpecTcl usually makes laying out a GUI much easier, 
there is one program that we saw that would be much harder to do
with SpecTcl, than in Motif or Tcl/Tk.  The reason is that in
SpecTcl, you have lay out the widgets individually.  Circle that
program, from the following list, which might extend to the next
page if this question is at the bottom of the page.

\begin{enumerate}
\item xbox.c
\item xdraw1.c
\item xdraw2.c
\item xfarewell.c
\item xgoodbye.c
\item xhello.c
\item xlots.c
\item xmove.c
\item xtext.c
\end{enumerate}

\ite{8} Here is a list of structs that X uses.  Following, is a 
list of names.  For each name, write the number of the struct
that this applies to.   You are not expected ever to have seen
the struct contents before.  The idea is to use your knowledge of 
what each data type does to find a struct whose components look
appropriate.

\begin{enumerate}
\item
\begin{verbatim}
typedef struct {
    short x, y;
} 
\end{verbatim}

\item
\begin{verbatim}
typedef struct {
	unsigned long pixel;
	unsigned short red, green, blue;
	char flags;  /* do_red, do_green, do_blue */
	char pad;
} 
\end{verbatim}

\item
\begin{verbatim}
typedef struct {
	int type;		/* of event */
	unsigned long serial;	/* # of last request processed by server */
	Bool send_event;	/* true if this came from a SendEvent request */
	Display *display;	/* Display the event was read from */
	Window window;	        /* "event" window it is reported relative to */
	Window root;	        /* root window that the event occured on */
	Window subwindow;	/* child window */
	Time time;		/* milliseconds */
	int x, y;		/* pointer x, y coordinates in event window */
	int x_root, y_root;	/* coordinates relative to root */
	unsigned int state;	/* key or button mask */
	unsigned int button;	/* detail */
	Bool same_screen;	/* same screen flag */
} 
\end{verbatim}

\item
\begin{verbatim}
typedef struct {
	int function;		/* logical operation */
	unsigned long plane_mask;/* plane mask */
	unsigned long foreground;/* foreground pixel */
	unsigned long background;/* background pixel */
	int line_width;		/* line width */
	int line_style;	 	/* LineSolid, LineOnOffDash, LineDoubleDash */
	int cap_style;	  	/* CapNotLast, CapButt, 
				   CapRound, CapProjecting */
	int join_style;	 	/* JoinMiter, JoinRound, JoinBevel */
	int fill_style;	 	/* FillSolid, FillTiled, 
				   FillStippled, FillOpaeueStippled */
	int fill_rule;	  	/* EvenOddRule, WindingRule */
	int arc_mode;		/* ArcChord, ArcPieSlice */
	Pixmap tile;		/* tile pixmap for tiling operations */
	Pixmap stipple;		/* stipple 1 plane pixmap for stipping */
	int ts_x_origin;	/* offset for tile or stipple operations */
	int ts_y_origin;
        Font font;	        /* default text font for text operations */
	int subwindow_mode;     /* ClipByChildren, IncludeInferiors */
	Bool graphics_exposures;/* boolean, should exposures be generated */
	int clip_x_origin;	/* origin for clipping */
	int clip_y_origin;
	Pixmap clip_mask;	/* bitmap clipping; other calls for rects */
	int dash_offset;	/* patterned/dashed line information */
	char dashes;
} 
\end{verbatim}


\end{enumerate}

\begin{enumerate}

\itemsp GC
\itemsp XButtonEvent
\itemsp XColor
\itemsp XPoint

\end{enumerate}

\ite{5} For each of the following properties, decide whether it
applies more to the basic C character strings, or to Motif
compound strings.  Write a \textbf{C} or an \textbf{M}, as
appropriate, before each statement.

\begin{enumerate}
\itemsp The data type is XmString.
\itemsp The usual routine that catenates two of it modifies its first argument.
\itemsp It makes GUIs that function in other languages easier to write.
\itemsp The routines that manipulate it allocate their own storage.
\itemsp The number of bytes that a variable of this type occupies is less
likely to correspond to the number of characters in the string.
\end{enumerate}



\item Consider the following X program:

\lgrindfile{xexam}

\begin{enumerate}

\ite{4} What does this program do?
\asp{2}

\ite{2} Sketch the tree of widgets from the top level widget to
the leaves.
\asp{3}

\ite{2} How many push button widgets are there?  Which routine is called
when each is clicked on?
\asp{1}

\ite{2} What are the labels that will print in the command widgets by default?
\asp{1.5}

\ite{1} What file where in your account will resources be read from (assuming that
things are configured as I've described in class)?
\asp{.5}

\ite{1} Suppose you want to change the label printed inside (only) the first
leaf widget.  Give a possible line to add to the resources
file.
\asp{.5}

\ite{1} Suppose you run the program thus: {\tt xexam -bg red}.
Where, if anywhere, in your program is the bg resource recognized?
\asp{.5}

\end{enumerate}



\ite{1}  How can your X program tell the user ran it with
command-line arguments that are not X resources and values?
\asp{.5}

\ite{1} In \texttt{xgoodbye.c}, why did I place the \texttt{Quit} routine before
\emph{main}?  For reference, here is part of that program, w/o comments.

\begin{verbatim}
void Quit(w, client_data, call_data)
    Widget  w;
    XtPointer client_data, call_data;
{   printf("It was nice knowing you.\n");
    exit(0); }

void main(int argc, char **argv)
{   Widget  top_level, goodbye_widget;
    XtAppContext app;
    top_level = XtVaAppInitialize(&app, "XGoodbye", NULL, 0, &argc, argv, NULL, NULL);
    goodbye_widget = XtVaCreateManagedWidget("goodbye", xmPushButtonWidgetClass, 
						top_level, NULL);
    XtAddCallback(goodbye_widget, XmNactivateCallback, Quit, 0);
    XtRealizeWidget(top_level);      
    XtAppMainLoop(app);      
}
\end{verbatim}
\asp{.5}



\ite{1} In \texttt{xlots.c}, what are the names of the widgets
created with the following code:
\begin{verbatim}
for (i = 0; i < N; i++)
   {   sprintf (s, "%d", i);
       w[i] = XtVaCreateManagedWidget (s, xmLabelWidgetClass, bb, NULL);
   }
\end{verbatim}
\asp{.5}

\ite{2} What are the names of the widgets
created with the following code:
\begin{verbatim}
char *a[]={"Sun", "DEC", "IBM", "MS"};
char s[100];
char *t;

for (i = 0; i < N; i++)
   {   sprintf (s, "%d", i);
       t=strcat(s,a[i]);
       w[i] = XtVaCreateManagedWidget (s, xmLabelWidgetClass, bb, NULL);
   }
\end{verbatim}
\asp{.5}


\ite{1} Program \texttt{xbox.c} has the following code:

\begin{verbatim}
/* PopupDialog:  callback for the PRESSME button. */
void PopupDialog(w, client_data, call_data)
    Widget  w;
    XtPointer client_data, call_data;
{
    Widget  parent;
    static Widget dialogbox = (Widget) NULL;
    parent = (Widget) client_data;

    if (!dialogbox) {  /* Create this only the first time thru. */
	dialogbox = XmCreatePromptDialog(parent, "dialogbox", NULL, 0);	
    }
    XtManageChild(dialogbox);
}
\end{verbatim}
How is it that the \texttt{if} test creates the dialogbox only the
first time thru?
\asp{.6}



\ite{2} There is something unusual about the identity of the
widget returned when you call \texttt{XmCreatePromptDialog}, in
contrast to most of the other widget creation routines.  What? 
\asp{.6}



\ite{2} In \texttt{xfarewell.c}, what is the following doing?
\begin{verbatim}
  static XtActionsRec my_actions[] =
  {
    {"confirm", (XtActionProc) Confirm},
    {"quit", (XtActionProc) Quit},
  };
  ...
  XtAppAddActions (app_context, my_actions, XtNumber (my_actions));
\end{verbatim}
\asp{1}


\ite{1} Various programs have had code like the following to set a
widget's label:
\begin{verbatim}
    s2=XmStringCreateSimple("newlabel");
    XtVaSetValues (but, XmNlabelString, s2, NULL);
\end{verbatim}
Since I can set a widget's label in a resource file w/o going thru
this mess, 
why can't I just say this in the program?
\begin{verbatim}
    XtVaSetValues (but, XmNlabelString, "newlabel", NULL);
\end{verbatim}
\asp{.6}


\ite{1} In \texttt{xdraw2.c}, why did I realize the top level
widget before creating the graphics contexts?
\asp{1}


\ite{1} In \texttt{xtext.c}, what does the following code
accomplish:
\begin{verbatim}
XtUnmanageChild(XmMessageBoxGetChild(help_widget, XmDIALOG_CANCEL_BUTTON));
\end{verbatim}
\asp{.7}


\ite{1} What is the purpose of a cascade button on a menu bar?
\asp{.7}


\ite{3}
What is a resource in X?  Name 4 resources.
\asp{1.5}

\ite{2}  What does it mean to \emph{manage} a widget?
\asp{1}


\ite{2}
When you call {\tt XtMainLoop()} your program loses control to the
X system.  Name two ways for it to get back control later (without
terminating the X system)?
\asp{1}


\item
\begin{enumerate}

\ite{1}
Why might you want to have the same callback routine for more than one
widget?
\asp{.5}

\ite{1}
In the above case, name one way for the routine tell which widget
it is responding to?
\asp{1}
\end{enumerate}


\ite{2} You can pass one variable to the callback routine, in the client
data field.  Suppose that you wish to make two variables used in
the main program to be available in the callback routine.  Give
two different ways to do it.
\asp{1}



\ite{1} When does a widget not have a window associated with it?
\asp{.7}


\ite{4}
For each of the following routines, say whether it is an
\emph{Xlib} routine, an \emph{Xt} routine, or a \emph{Motif}
routine:
\begin{enumerate}
\setlength{\itemsep}{0in}
\item XtVaAppinitialize
\item XmStringCreateSimple
\item  XtVaSetValues
\item XFillArc
\end{enumerate}
\asp{.6}


\item Graphics contexts:
\begin{enumerate}
\ite{1} How does a graphics context lower load on the net?
\asp{1}


\ite{2} Name 3 attributes stored in a GC.
\asp{1}
\end{enumerate}


\ite{2} One Motif design principle is called
\emph{I18N}. What does this mean?  Name one feature that we've
seen that supports this. 
\asp{.8}

\ite{2} What's wrong with this C program:
\begin{verbatim}
char command[4]="ls ";
char arg[5]="/tmp";
char line[100];

line=strcat(command,arg);
system(line);
\end{verbatim}
\asp{1}


\ite{1} 
Why is it advantageous for the several clients on one graphics
server to use the same colors, instead of each using slightly
different colors?
\asp{1}

\ite{1}
Suppose that one program wants to use 250 colors on its own.  How
might it do something with color maps to lessen the impact on
others clients?
\asp{1}


\ite{1}  It's clear why 8-bit frame buffers use color maps (i.e.,
to select which 256 colors of the $2^{24}$ possible ones to use).
However, some 24-bit buffers also use color maps.  (It's not one
$2^{24}$ entry cmap, but 3 256-entry cmaps, one per primary
color.)  Why might they want to do this since it does make things
more complicated?
\asp{.7}


\item Convenience widgets:
\begin{enumerate}
\ite{1} What are they?
\asp{.7}
\ite{1} Name one.
\asp{.5}
\end{enumerate}


\ite{1} Suppose that you create an information dialog widget, but
don't want the cancel button to appear.  How can you effect this?
\asp{.6}


\ite{1} How does a program know which file to read to get the
resources?

\ite{1} What does \verb+XtAppMainLoop+ do?

\ite{2} What does \verb+XtRealizeWidget+ do?  It really does
several things; I'll take any one.
\asp{.5}


\ite{1}
Why does X use special data types, like \verb+Point+, instead of just
saying \verb+struct { int x,y }+?
\asp{.5}


\ite{1}
Why does X use special data types, like \verb+Position+, instead of just
saying \verb+int+?
\asp{.5}


\ite{4} Give two advantages and two disadvantages of Motif compared to the
Athena widgets.
\asp{1}


\ite{2}
In your C program, Motif uses the preprocessor symbol \texttt{XmNx}
to represent the resource that would be named {\em x} in your
resource file.   How does this help to catch typos in your program?
\asp{1}


\item Motif uses compound text strings.
\begin{enumerate}
\ite{1} What are they?
\ite{2} Give 2 advantages of them.  The following is not
acceptable.
\begin{quote}
``It's harder to debug compound strings, so you use more CPU time.
Motif comes from the OSF, of which IBM is a partner.  So, compound
strings help IBM to sell more computers.''
\end{quote}
\end{enumerate}
\asp{1.3}

\item
The resource file for \verb+charset.c+ said this:

\begin{verbatim}
*pb1.fontList: -*-courier-*-r-*--12-*=charset
*pb2.fontList: -*-courier-*-r-*--14-*=charset
*pb3.fontList: -*-courier-*-r-*--18-*=charset
\end{verbatim}
\asp{1.2}


\begin{enumerate}
\ite{2} What is this \verb+charset+ used for and where in the
program might I refer to it?

\ite{1} Why is it not a problem for the three widgets all to use
the same \verb+charset+?  That is, won't the last use of
\verb+charset+ override the previous two?
\end{enumerate}


\ite{2} In \texttt{xcharset.c}, how can the 3 widgets' different
resources cause their labels to have different type fonts, when
labels are exactly the same compound char string?  FYI, here is
the relevant part of the program:
\begin{verbatim}
    XmString      text;

    toplevel = XtVaAppInitialize(&app, "XCharset", NULL, 0,
        &argc, argv, fallbacks, NULL);
    text = XmStringCreateSimple("Testing, testing...");
    rowcol_widget = XtVaCreateManagedWidget("rowcol",
        xmRowColumnWidgetClass, toplevel, NULL);
    XtVaCreateManagedWidget("pb1", xmPushButtonGadgetClass, rowcol_widget,
        XmNlabelString, text, NULL);
    XtVaCreateManagedWidget("pb2", xmPushButtonGadgetClass, rowcol_widget,
        XmNlabelString, text, NULL);
    XtVaCreateManagedWidget("pb3", xmPushButtonGadgetClass, rowcol_widget,
        XmNlabelString, text, NULL);
\end{verbatim}
\noindent and the resource file:
\begin{verbatim}
*pb1.fontList: -*-courier-*-r-*--12-*=charset
*pb2.fontList: -*-courier-*-r-*--14-*=charset
*pb3.fontList: -*-courier-*-r-*--18-*=charset
\end{verbatim}
\asp{1}


\ite{1}
Name one programming error wrt compound strings that can lead
to a storage leak.
\asp{.6}


\ite{1}
One compound string can have several fonts.  Different X servers
can have different fonts available.  This would seem to suggest
that there might be a problem with a compound string referring to
a font that doesn't exist on some servers.  Name one feature in
the compound string format that ameliorates this possible problem.
\asp{.6}


\ite{1} What is a \emph{font server}?
\asp{.5}


\ite{1} Compound strings usually should be freed when you are
finished with them.  What happens if you don't?


\ite{3} Name 3 differences between standard C character strings
and Motif compound text strings.


\ite{2} Suppose that we wish to put 3 label widgets and 5 command
widgets in a popup.  How can we do this, given that a transient shell
widget can have only one child?
\asp{.5}


\ite{2} Consider the following widget definition:
\begin{verbatim}
jambo=XtVaCreateManagedWidget("gday",labelWidgetClass,box,NULL);
\end{verbatim}
and the following fragment from a resource file:
\begin{verbatim}
*jambo.command: Yes
*jambo.labelString: No
*gday.command: Maybe
*gday.labelstring: bon jiorno
*end.label: Freeze!
*Command.labelString: Finish!
*labelString: Default.
*label: Itty Bitty Machines
\end{verbatim}

What will be the label on this widget?
\asp{.5}


\ite{2} Consider the following widget definition in a program
whose class is \texttt{XExam}
\begin{verbatim}
jambo=XtVaCreateManagedWidget("gday",labelWidgetClass,box,NULL);
\end{verbatim}
and the following fragment from a resource file:
\begin{verbatim}
*jambo.command: Yes
*jambo.labelString: No
*gday.command: Maybe
*gday.labelstring: bon jiorno
*end.label: Freeze!
*Command.labelString: Finish!
*labelString: Default
*label: Itty Bitty Machines
XExam*labelString:  Forty two
\end{verbatim}

What will be the label on this widget?
\asp{.5}


\ite{2}
Sometimes you see a line such as this:
\begin{verbatim} 	XtSetArg(arg[i], XtNwidth, &width);    i++;
\end{verbatim}
but sometimes it is written thus:
\begin{verbatim}	XtSetArg(arg[i], XtNwidth, width);    i++;
\end{verbatim}
Why is {\tt width} passed by value sometimes and by address at others?

\ite{2}  What is the difference between an event and an action?
\asp{1}


\item  Events fall into two broad categories.
\begin{enumerate}
\ite{2} What are they?
\asp{.6}
\ite{2} Give an example of each.
\asp{.6}
\end{enumerate}

\ite{1} How do you establish the relation between an event and an
action?
\asp{.5}

\ite{1} There are events that are user inputs, and then there are
events that are system status changes.  Name one of the latter.
\asp{.6}


\ite{3} Three types of manager widgets are \emph{rowColumn},
\emph{form}, and \emph{bulletinBoard}.  Distinguish between them.
\asp{1}


\ite{1}
In the mainwindow widget menu bar, what's the purpose of declaring
one button to be the help button?
\asp{1}


\item Window managers:
\begin{enumerate}
\ite{2} Name 2 things that a window manager does.
\asp{1}

\ite{2} If you kill the window manager, what changes in the
appearance on the screen?  What functionality do you lose?
\asp{1}

\ite{1} Name any one window manager in general use on RCS.
\asp{.6}
\end{enumerate}


\ite{1}
Resources are grouped into classes, just as programs are.  E.g,
\texttt{foreground} and \texttt{bottomShadowColor} are both class
\texttt{Foreground}. However, the background resource is in a
different class, \texttt{Background}.  It would seem more logical
to group all color-related resources into one class called
\texttt{Color}.  Why do think that the OSF did not do this?
\asp{1}



\ite{2} While browsing the resources file for \verb+Netscape+, I
found this piece of a translation table.
\begin{verbatim}
*globalTranslations:                    #override                       \n\
        Meta ~Ctrl<Key>A:               addBookmark()                   \n\
         Alt ~Ctrl<Key>A:               addBookmark()                   \n\
        Meta ~Ctrl<Key>B:               viewBookmark()                  \n\
         Alt ~Ctrl<Key>B:               viewBookmark()                  \n\
        Meta ~Ctrl<Key>C:               copy()                          \n\
         Alt ~Ctrl<Key>C:               copy()                          
\end{verbatim}
Which are the actions, and which the events here?
\asp{1}

\ite{2} While browsing the resources file for \verb+xcalc+, I
found this piece of a translation table.
\begin{verbatim}
*hp.bevel.screen.LCD.Translations:	#replace\n\
				<Key>p:pi()\n\
				<Key>i:inverse()\n\
*ti.button5.Translations: #override<Btn1Up>:off()unset()\n\
			  <Btn3Up>:quit()
\end{verbatim}
Which are the actions, and which the events here?
\asp{1}


\item Look at this line from \verb+xtext.c+.
\begin{verbatim}
XtVaSetValues(clear, XmNleftAttachment, XmATTACH_WIDGET,
	      XmNleftWidget, 	print, NULL);
\end{verbatim}
\begin{enumerate}
\ite{1} What does the \verb+Va+ mean as part of the routine name?
\asp{.7}
\ite{1} This routine claims that it is setting values.  What sort
of values.  E.g., is it assigning C variables?
\asp{.7}
\ite{1} What is the approximate effect of this values setting?
\asp{.7}
\end{enumerate}


\ite{1} How do you establish the relation between a routine in your
program and an action?


\item The designers of X made some choices about the capabilities
of a graphics server, and picked a middle road.
\begin{enumerate}
\ite{2} Name one capability that they included in the server, which
they might have omitted.  Give an advantage of including it.
\asp{.6}
\ite{2} Name one capability that they omitted, which they might
have included.  Give a disadvantage of omitting it.
\asp{.6}
\end{enumerate}


\ite{1} Consider this code, which creates two widgets.

\begin{verbatim}
Widget but;
but = XtVaCreateManagedWidget("but",xmPushButtonWidgetClass,top_level,NULL);
but = XtVaCreateManagedWidget("but2",xmLabelWidgetClass,top_level,NULL);
\end{verbatim}

\noindent and this extract from the resource file.

\begin{verbatim}
*but.labelString: jambo!
\end{verbatim}

Does this affect the button widget, the label widget, both, or neither?
\asp{.5}


\item
For certain technical reasons, a few resources cannot be specified
in a resource file, but must be specified in the program.  I'm not
thinking of a reason like that it is undesirable to allow the
user access to it.  This resource just cannot be given a value in
the resource file, even if you, the programmer, wanted to.
\begin{enumerate}
\setlength{\itemsep}{0in}
\ite{1} Name one such resource.
\asp{.5}
\ite{1} Why cannot it be specified in the resource file?
\end{enumerate}
\asp{1}


\ite{1} Suppose that you create a widget, and then assign it to
another variable:

\begin{verbatim}
Widget hello, hello2;
hello = XtVaCreateManagedWidget("hello2",xmLabelWidgetClass,top_level,NULL);
hello2= hello;
\end{verbatim}

If you want to set a resource for the widget referenced by the C variable \texttt{hello}, should you do this:
\begin{verbatim}
*hello.labelString: jambo!
\end{verbatim}
or this:
\begin{verbatim}
*hello2.labelString: jambo!
\end{verbatim}
or does it not matter?
\asp{.5}


\ite{1} Suppose that you create a widget, and then assign it to
another variable:

\begin{verbatim}
Widget hello, hello2;
hello = XtVaCreateManagedWidget("hello",xmLabelWidgetClass,top_level,NULL);
hello2= hello;
\end{verbatim}

If you want to set a resource for hello2, should you do this:
\begin{verbatim}
*hello.labelString: jambo!
\end{verbatim}
or this:
\begin{verbatim}
*hello2.labelString: jambo!
\end{verbatim}
or does it not matter?
\asp{.5}

\ite{2} What's the difference between a rowcolumn widget, a paned
widget, and a menubar widget?


\ite{2}
Xlib, at least as we've seen it, counts coordinates in pixels.
How would you write a program to display the same on the different
CRTs at RPI, which have 3 different resolutions?  How would you
get any information that you propose to use?
\asp{.6}



\ite{1}
If your screen is 1024x1024, and has 8 bits per pixel, how much
memory does the frame buffer take?
\asp{.6}


\item
Suppose that you want to write a calculator program to run on both
the Suns, with $900\times1152$-resolution displays, and the X stations, with
$480\times640$-resolution displays.  You want the program with all its
buttons to fill 1/2 of the screen in either case.

\begin{enumerate}
\ite{2}    Would it be easier to hold
the buttons in a \emph{form} widget or a \emph{bulletinboard}
widget?  Why?
\asp{.6}

\ite{1} 
Why should you not use a \emph{rowcolumn} widget?
\asp{.5}
\end{enumerate}


\ite{2}
What's wrong with this?  Fix it.

\begin{verbatim}
char *c;
c[0]='h';
c[1]='i';
c[2]='\0';
printf("The message of the day is: %s\n",c);
\end{verbatim}
\asp{.5}


\item
Suppose that Bill is running on \texttt{whitehouse.gov} and he
wishes to start an \verb+xeyes+ program to display on Newt's
personal workstation, 
\texttt{house.gov}.
\begin{enumerate}
\ite{1} How does Bill tell xeyes to display on Newt's machine?
\asp{.5}
\ite{1} How does Newt permit Bill's program to display on his
machine?
\asp{.5}
\ite{1} Which is the client and which the server?
\asp{.5}
\end{enumerate}

\ite{2} What functionality does the \emph{Xt} level add to the X
window system?
\asp{.5}

\ite{2} Near the start of my programs, there are these lines:

\begin{verbatim}
#include <Xm/Xm.h>       /* Needed by all Motif programs. */
#include <Xm/Label.h>   /* This is for the Motif Label Widget */
\end{verbatim}

When you compile the program, how does the compiler know where to
look for the include files?
\asp{.7}

\ite{2} What is the difference between the client data and the
call data in a callback routine?

\ite{1} Name one type of information that we saw passed passed in the
\texttt{call\_data} argument to a callback routine.
\asp{.5}


\ite{1} Why does X make it so complicated to use new colors in Xlib?
Whether or not the MIT grad students who wrote much of X are sadistic is
not relevant.
\asp{1}


\ite{1} Consider the file
\texttt{/campus/X11/R5/\-core/1.0/\-@sys/lib/\-X11/rgb.txt}, which maps
from color names, like cyan, to values.
\begin{enumerate}
\ite{1}
We see lines like this:

\begin{verbatim}
255 228 181		moccasin
\end{verbatim}

What does 255 mean?
\asp{.5}

\ite{1}
There's another line like this:

\rule{1.2in}{.15in}\verb+   Blue+

Unfortunately the first half of the line is unreadable :-).  Tell
me what number(s) are hidden by the black box.
\end{enumerate}


\ite{2} In \texttt{xdraw2.c}, there are expressions like this:

\begin{center}
\texttt{XAllocColor(display, cmap, \&orange)}
\end{center}

\noindent Why is \texttt{cmap} passed directly, but the address of
\texttt{orange} is passed in?
\asp{.5}

\ite{1} Why might \texttt{XAllocColor} fail on a color display?
\asp{.5}


\ite{1}
An X server has no protection or hiding between its various
clients.  Name one program that we've seen that uses this.
\asp{.6}


\item
Client-server communications operate asynchronously.  
\begin{enumerate}
\ite{1} What does this mean?
\asp{.6}
\ite{2}Name an advantage and a disadvantage of this.
\asp{.6}
\end{enumerate}


\ite{1}
Name a widget class in the widget class hierarchy, is
subclassed from the label widget class
\asp{.6}


\ite{1} Normally the widget instance hierarchy matches the window
hierarchy.  Name an exception. 
\asp{.6}



\ite{6}
For each of the following  properties, say whether it applies more to Motif
and to Tcl/tk.

\begin{enumerate}
\item More industry standard.
\item Easier to use.
\item Faster to execute.
\item Funded by ARPA.
\item Funded by OSF.
\item Source code freely available.
\end{enumerate}

\item  Look at the following Tcl program.

\begin{verbatim}
#! /home/wrf/bin/wish -f

proc getdate { } {
  global date
  set datef [open "|date"]
  set date [read $datef] 
  close $datef
}

getdate

button .hello -textvariable date  -fg red -command getdate

pack .hello 
\end{verbatim}
 
\begin{enumerate}

\ite{1} What does \texttt{-textvariable} do?
\asp{.6}
\ite{1} What does \texttt{-command} do?
\asp{.6}
\ite{1} What does the program as a whole do?
\asp{.6}
\end{enumerate}




\ite{2} When the Bresenham line algorithm was invented, floating
point operations were much slower than fixed point.  However, on
current CPUs, floating point is generally as fast as fixed (maybe
faster!).   However, the Bresenham algorithm still is useful in
modern hi-performance graphics.  Why?
\asp{1}

\ite{2} What is the difference between \verb+http+ and
\verb+html+?

\ite{1} Which one of the following is correct?
\begin{verbatim}
Position x,y;
\end{verbatim}
\begin{enumerate}
\item \verb+    XtVaGetValues (top_level, XmNx, &x, XmNy, &y, NULL);+
\item \verb+    XtVaGetValues (top_level, XmNx, x, XmNy, y, NULL);+
\end{enumerate}

\ite{2}  What file is accessed by the following URL?

\verb+http://www.rpi.edu/~pipes/+

\item Look at the condensed class home page below.

\begin{center}
\footnotesize
\begin{boxitpara}{rectcartouche}
\begin{verbatim}
<head><title>35.475 Computer Graphics, Spring 1995, Index Page</title></head>

<BODY><P>

<H1>35.475 Computer Graphics</H1>
<hr>
<H3> When and Where</H3>
Lectures: MW 11am - noon in CC3051.  <br>
Lab: F, either 11-noon in VCC-N, or noon-1 in VCC-S.
</DL>
<hr>
<H3>Misc Pages</H3>
<MENU>
<LI> <A HREF="lectures.html">List of lectures</A>
<LI> <A HREF="register.html">Please register yourself.</A>
<LI> <A HREF="comment.html">Give us your comments or suggestions.</A>
<LI> <A HREF="class_list.html">Class List.</A>
</MENU>
<hr>

<H3>Handouts and Old Homeworks</H3>

My secretary, Audrey Hayner, in JEC 6012, has copies of most old
handouts, and of old graded homeworks that were not picked up in
class.

<hr>
<i>Last updated: 27 Feb 95</i>
<hr>
<P><ADDRESS>
<A HREF="http://www.ecse.rpi.edu/~wrf/">
Wm. Randolph Franklin</A>, ECSE Dept, Rensselaer Polytechnic
Institute, wrf@ecse.rpi.edu
</ADDRESS>
</BODY>
\end{verbatim}
\end{boxitpara}
\leavevmode % w/o this get a latex error
\end{center}

\begin{enumerate}
\ite{1} What does \verb+<h3>+ mean?
\ite{1} What does \verb+<hr>+ mean?
\ite{3} List the last URL seen on the page.  What machine is the
page retrieved from?  What is the pathname of the file on that machine?
\end{enumerate}
\end{enumerate}



\setcounter{questionnumber}{\value{enumi}}

\subsection*{$\overline{X}$ questions}

\begin{enumerate}
\setcounter{enumi}{\value{questionnumber}}

\item About the Salim Abi-Ezzi patent:
\begin{enumerate}
\ite{1} What does {\bf NURBS} stand for?
\ite{2} The {\bf R} represents a choice that was made.  What was
the alternative, and why was the {\bf R} choice better?
\end{enumerate}

\item About {\em Foley}, chap 1:
\begin{enumerate}
\ite{1} Why is raster graphics better than vector graphics?
\ite{1} So, why didn't people use raster graphics a lot 20 years
ago?
\end{enumerate}

\ite{4} Calculate the pixels on a circle of radius 19 using the
Bresenham method.  Show your work.

\ite{3} What are bundled attributes?  Give an advantage and a
disadvantage of them.

\ite{3} When filling a polygon, which is defined by a list of vertices,
the simple way, what problem is caused by the current scan line going
through a vertex?  How do you work around this problem?

\ite{1} Portability has at least one disadvantage, other than making it
easier for your competitor to copy your program.  Name it.

\ite{1} What is the advantage of a Bresenham-style algorithm compared to
simpler, more obvious methods?

\ite{1} Why do graphics packages contain polyline routines in spite of
the fact that users could easily just call the line routine many times?

\ite{2} Name an advantage and a disadvantage of specifying attributes
for, e.g., a line, in every call to draw the line.

\ite{2}
Suppose you used a paint program to outline a region by setting pixels
around its border.
Describe how the paint program could operate to fill its interior.
Illustrate this by filling the interior of the letter {\LARGE B}.

\ite{2}
How can the above method be used when colorizing movies?  The problem is
that a frame of a black-and-white movie doesn't have pixels of a certain
value outlining each region that should be a single color.

\ite{1}  Why would you be interested in doing a Bresenham algorithm at a
sub-pixel accuracy instead of simply rounding the line endpoints (or
circle center and radius) to the nearest pixel before applying
Bresenham?

\ite{1}  Why would you be interested in doing a Bresenham algorithm at a
sub-pixel accuracy instead of simply using a higher resolution device?

\ite{2} Write your name clearly on the top of your answer sheet.

\item Consider the program \verb+coloredit.c+.  Ignore any
Motif features that are new to you, except for the ones summarized
below; understanding the others is not necessary for answering this
question.


\lgrindfile{coloredit}
\ \

\noindent \emph{Summary of new routines}
\begin{itemize}

\item \texttt{XtInitialize} returns a top level widget, like
\texttt{XtVaAppInitialize}.
\item \texttt{XtCreateManagedWidget} and \texttt{XtCreateWidget} are like
\texttt{XtVaCreateManagedWidget}; their relevant args are the first three: name,
widget types, parent widget.
\end{itemize}
\begin{enumerate}
\ite{1} What is the program's class?
\ite{4} Sketch the tree of widget instances.
\ite{3} Name the three row column widgets.
\ite{5} Name the label widgets.
\end{enumerate}

\ite{1} Correct this program.
\begin{verbatim}
main()
   {
        char *s;
        s[0]='h';
        s[1]='i';
   }
\end{verbatim}

% \ite{0} What does this C program in Figure \vref{f:schnitzi} do?

% \begin{figure}[!htb]
% \footnotesize
% \begin{center}
% \begin{verbatim}
% #include <stdio.h>
% int r=0,x,y=0,    /*
% nt                 /
% c
% lx  /\ / \ /|\ /|\
% u,  \/ \\/ \-/ \ /
% dn                        ]
% e= ||| /|\  /\  /\        =                     p p
%  0  \/ \ / |||  \/        f        f       f    r r
% <,                        tw       t       s s  i i
% sy  || /|\      /\     i  eh       e       e c  n n
% t= /-  \ /     |||     n  li       l       e a  t t
% d0     |||             t  ll       l  {<   k n  f f
% i,  ||     /|\ /|\        (e s     (  fr   ( f  ( (
% o    _ ||\ \ / \-/     tu s( t    usw o; i s (  " "
% .   || - /           c [[ tg rv   [th rx=f t "  % \
% h                   mh111 de{l)   +di (+'( d+%  c n+
% >                   aa000 itte; > +il x+ n inc  " "+
%  /                  ir000[ns[nytyrrne<=)'<xn,"  , );
%  */                 n         =             0     ;
%                main        (       )   {
%                 char       v   [
%                 100    ]                   ,  s;
%           int t[100     ]                    ,
%               u[100               ]     ;  u
%                   [                   0
%       ]=ftell(stdin        )       ;
%          while(gets (v    )                 )
%                  {t  [         r ]=
%              strlen                 (
%                v);y=
%                  t[         r             ]
%                  >y             ?        t [
%                   r         ]    :   y;
%               u[++r   ]     =
%         ftell(stdin )   ;                          }
%               while          (                    n
%                   <              y        )
%             {for(x=0             ;         x
%             <r;x++) {                           s
%                =' '       ;
%               if(n<             t        [
%                   x            ]     )    {
%         fseek(stdin    ,u       [     x    ]
%                  +n   ,  0 )                     ;
%         scanf("%c",                          &
%                                        s        )
%                    ;                             }
%        printf("%c",   s                       )
%                                             ;      }
%       printf("\n");                 n
%                  ++   ;            }           }
% \end{verbatim}
% 
% \leavevmode % w/o this get a latex error
% \caption{schnitzi.c}
% \label{f:schnitzi}
% \end{center}
% \end{figure}

\ite{4} Print your name legibly on the top of page one, if you
haven't already.

\end{enumerate}


{\tt Total: \thepoints\ points}

{\em End of exam} \label{end}

\vfill\hfill {\footnotesize \today, \printtime\ /dept/ecse/graphics/ho\handout.tex}
\end{document}

% Emacs vars
% Local Variables:
% eval: (outline-minor-mode)
% outline-regexp: "\\\\sec\\|\\\\subs\\|% Em"
% font-lock-keywords-p: t
% font-lock-keywords: (("\\(\\\\\\w+\\)\\W" 1 font-lock-keyword-face t) ("\\(\\\\\\w+\\){\\([-a-zA-Z0-9 .?]+\\)}" 2 font-lock-function-name-face t) ("\\(\\\\\\w+\\){\\(\\w+\\)}{\\(\\w+\\)}" 3 font-lock-function-name-face t) ("\\(\\\\\\w+\\){\\(\\w+\\)}{\\(\\w+\\)}{\\(\\w+\\)}" 4 font-lock-function-name-face t) ("{\\\\em\\([^}]+\\)}" 1 font-lock-comment-face t) ("{\\\\bf\\([^}]+\\)}" 1 font-lock-keyword-face t) ("^[ 	^j]*\\\\def[\\\\@]\\(\\w+\\)\\W" 1 font-lock-function-name-face t) ("[^\\\\]\\$\\([^$]*\\)\\$" 1 font-lock-string-face t))
% eval: (font-lock-fontify-buffer)
% End:



